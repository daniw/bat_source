\section{Introduction}
\todo[inline]{Introduction to the project, reasoning behind, $\ldots$}

\subsection{Definitions}

\subsubsection{Units}
\ac{SI} units are used in this document. 

\FloatBarrier

\subsubsection{\acs{I2C} address}
The adress of \ac{I2C} devices consists of 7 bit. On the bus, the R/W bit is appended after the \ac{LSB} of the address, which shifts the address to the left by one bit. This is shown in \autoref{fig:intro_i2c_address}. 

\begin{figure}[h!]
    \centering
    \begin{tikztimingtable}[
        % For labelled L, see:
        % https://tex.stackexchange.com/questions/37254/tikz-timing-text-labels-in-symbols-other-than-d-and-time-axis-discontinuities
        timing/metachar={{K}[2]{#1l !{++(0,+.5\yunit)} N[rectangle,scale=.6]{#2} !{++(0,-.5\yunit)} #1l}},
        timing/metachar={{J}[2]{#1h !{++(0,-.5\yunit)} N[rectangle,scale=.6]{#2} !{++(0,+.5\yunit)} #1h}},
        ]
        SDA & 2H4K{Start}4D{A6}4D{A5}4D{A4}4D{A3}4D{A2}4D{A1}4D{A0}4D{R/W}4K{Stop}2H \\
        SCL & 5H9{2L2H}3H \\
    \end{tikztimingtable}
    \caption{\acs{I2C} address}
    \label{fig:intro_i2c_address}
\end{figure}

In this documentation each \ac{I2C} address is represented without the R/W bit and right-aligned (\ac{LSB}-aligned). 

\FloatBarrier

\subsection{Directory structure}
\dirtree{%
.1 bat\_source/\DTcomment{Project root}.
.2 doc/\DTcomment{Documentation}.
.3 appendix/\DTcomment{Appendix}.
.3 bib/\DTcomment{Bibliography}.
.3 img/\DTcomment{Graphics and images}.
.3 layout/\DTcomment{Package to define document layout}.
.3 obj/\DTcomment{Documentation output}.
.3 src/\DTcomment{Documentation source}.
.3 test/\DTcomment{Test protocol}.
.2 hw/\DTcomment{Hardware}.
.3 Bat\_Source\_Assembly/\DTcomment{Multiboard assembly}.
.3 PS1-1/\DTcomment{Mainbard}.
.3 PS1-2/\DTcomment{User interface PCB}.
.3 PS1-3/\DTcomment{Programming adapter}.
.3 PS1-100/\DTcomment{Transformer connection PCB}.
.3 PS1-101/\DTcomment{Winding PCB - \qty{1}{\wdg}}.
.3 PS1-102/\DTcomment{Winding PCB - \qty{2}{\wdg}}.
.3 PS1-124/\DTcomment{Winding PCB - \qty{24}{\wdg}}.
.3 sim/\DTcomment{Scripts and simulations}.
.2 mech/\DTcomment{Mechanics}.
.2 sw/\DTcomment{Software}.
}
\todo{Software directory structure}
\acresetall
