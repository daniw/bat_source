\section{Concept}
\label{sec:concept}
%\lipsum

\subsection{Charging}
The user must be able to recharge the internal energy source. Multiple concepts are discussed: 

\begin{itemize}
    \item Integrated power supply with mains connector
        \\
        The device is connected to mains for charging. The integrated power supply provides the protective insulation. 
    \item Charging through output plugs
        \\
        A DC voltage is applied to the output pins when the device is switched of, which powers the charger. No insulation is necessary, because no electrically conductive parts are exposed. 
    \item Wireless charging
        \\
        The charger is powered with an alternating magnetic field that is applied from outside the device enclosure, similar to Qi chargers for phones. This might lead to a increased heat output when charging inside the storage enclosure. 
    \item External power supply with integrated charger
        \\
        An external power supply provides the power for charging. Depending on the selected connector type, protective insulation on the charging input or output port might be required. 
\end{itemize}

\subsection{Energy source}

\begin{itemize}
    \item Lead
        \\ Heavy, low power density, availability questionable due to new european battery regulation
    \item NiMH
        \\
    \item LiIon
        \\
    \item LiFePO4
        \\
\end{itemize}

\ReqRef{R:Bat_Change} requires the battery to be replacable without the need of special tools. For PCB-mounted batteries, a socket should be used, as soldering is considered as using special tools. 

\subsection{User interface}
The user must be able to set the operation mode and adjust the output current. The output voltage and current should be shown to the user. \\
If the cover of the enclosure is transparent, the display can be placed inside the enclosure, which allows for an inherent water proof construction. 

\subsubsection{User interaction}
\begin{itemize}
     \item Touch control
        \\
        Multiple touch sensors can be implemented that can be actuated through the enclosure. No tactile feedback is provided unless an additional actuator is implemented. 
    \item Hall sensors with external magnet
        \\
        Hall sensors can be integrated inside the enclosure which can be actuated by a magnet from outside the enclosure, which would inherently provide ingress protection. A magnet is needed to operate the device, which must always be kept close to the device and there is no tactile feedback as well. 
    \item Buttons / switches
        \\
        Depending of the type of button / switch selected, this provides feedback to the user. The buttons / switches must include ingress protection as they protrude through the enclosure. 
    \item Rotary encoder
        \\
        A rotary encoder with integrated push-button can be used. This solution will provide feedback to the user. The encoder must include ingress protection for water. 
\end{itemize}

\subsubsection{Display type}
\begin{itemize}
     \item 7-segment
        \\
    \item 13-segment
        \\
    \item Alphanumeric
        \\
    \item Dot marix
        \\
    \item Graphical
        \\
\end{itemize}

\subsubsection{Display technology}
\begin{itemize}
     \item \acs{LED}
        \\
        A \ac{LED} display is self-illuminating and therefore especially good readable in low-light conditions. For readability in sunlight, the brightness of the display must be sufficiently high. An automatic adjustment of the brighness might be beneficial. 
    \item \acs{LCD}
        \\
        \acp{LCD}
    \item \acs{TFT}
        \\
    \item \acs{OLED}
        \\
    \item \acs{VFD}
        \\
    \item E-paper display
        \\
\end{itemize}

\subsubsection{Calibration}
\ReqRef{R:Calibration} requires a means to calibrate the output values of the source. Therefore a communication interface must be implemented to allow sending and receiving commands and measurement values. 

\begin{itemize}
    \item Wired connection
        \\
        A wired conections (\ac{USB}, RS232, Ethernet) allows easy connection to the device, but might require additional galvanic isolation between the interface and the output terminals. 
    \item Optical
        \\
        An optical interface (eg. IRDa) allows for wireless communication with inherent galvanic insulation. As it requires line-of-sight to the device, it can not be accessed from large distances or connected to the wrong device, when handling multiple devices. 
    \item Radio
        \\
        Radio communication (Wifi, Bluetooth, \ldots) allows for wireless communication with inherent galvanic insulation. The range allows for a very flexible setup for calibration. 
    \item Magneto-optic
        \\
        An assymmetric interface with a magnetic field for one direction and optical communication for the oposite direction allows for wireless communication with inherent galvanic insulation. As this would be a custom interface, the implementation will require more effort. The magnetic field might interfere with the measurements during calibration. 
\end{itemize}

\subsection{Mechanical Design}

\FloatBarrier
\subsection{EMI Concept}

\FloatBarrier
\subsection{Interfaces}

