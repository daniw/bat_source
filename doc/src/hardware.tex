\section{Hardware}

\FloatBarrier

\subsection{Overview}
\label{sec_hw_overview}
\begin{itemize}
    \item Block diagram
\end{itemize}

\FloatBarrier

\subsection{Battery}
\label{sec_battery}
\begin{itemize}
    \item Cell selection
    \item Cell connection
    \item Cell mounting
\end{itemize}

\subsection{Battery Management System}
\label{sec_bms}
\begin{itemize}
    \item Voltage monitoring
    \item Current monitoring
    \item Temperature monitoring
\end{itemize}

\FloatBarrier

\subsection{Converter}
\label{sec_converter}
\begin{itemize}
    \item Topology
    \item Converter design
    \item Component selection
    \item Transformer design
    \item Secondary winding relay with switching DC link capacitors
    \item Charging operation
\end{itemize}

\FloatBarrier

\subsection{Voltage measurement}
\label{sec_volt_meas}
\begin{itemize}
    \item Voltage divider
    \item Differential amplifier
\end{itemize}

\FloatBarrier

\subsection{Current measurement}
\label{sec_cur_meas}
\begin{itemize}
    \item Shunt selection
    \item Amplifier selection
\end{itemize}

\FloatBarrier

\subsection{Output terminals}
\label{sec_out_term}
\begin{itemize}
    \item Common mode choke
    \item Fuse
    \item Reverse polarity protection
\end{itemize}

\FloatBarrier

\subsection{On/off controller}
\label{sec_onoff}
\begin{itemize}
    \item On
    \item Off
\end{itemize}

\FloatBarrier

\subsection{Microcontroller}
\label{sec_microcontroller}
\begin{itemize}
    \item Microcontroller selection
    \item GPIO assignment
    \item Analog reference
\end{itemize}
The STM32F334series is selected as microcontroller. This controller is suited for controlling a DC/DC converter due to the inclusion of the \ac{HRTIM}, which allows \ac{PWM} generation with a resolution as low as \qty{217}{\pico\second}. The QFP-48 package is selected due to the solderability and number of \ac{GPIO} pins. For the prototype the mximum of \qty{64}{\kibi\byte} is selected. This leads to the selected device: STM32F334C8T

\subsubsection{Pinout and module usage}
\paragraph{Reference voltage}
The internal reference voltage is only directly accessible in the BGA package. To access the internal voltage reference, the output of the operational amplifier (OPAMP2\_VOUT) is used. For this the amplifiert must be configured to output the refernce voltage. (OPAMP2\_CSR: TSTREF) the referance voltage is then accessible on PA6. 

\subsubsection{Memory}
To store persistent data such as serial number or calibration data, an \ac{I2C} \ac{EEPROM} is used. 

\FloatBarrier

\subsection{User interface}
\label{sec_ui}
\begin{itemize}
    \item Display
    \item Buttons
    \item Rotaty encoder
\end{itemize}

\FloatBarrier

\subsection{Power supply}
\label{sec_power_supply}
\begin{itemize}
    \item \qty{3.3}{\volt} supply
    \item Alternative charging power supply
\end{itemize}

\FloatBarrier

\subsection{Testing}
\label{sec_testing}
\begin{itemize}
    \item Automatic testing
    \item Testpoints
    \item Calibration
\end{itemize}

\FloatBarrier

\subsection{Modifications from index AA to index AB}

\subsection{Modifications from index AA to index BA}
